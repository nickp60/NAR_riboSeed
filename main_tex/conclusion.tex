Demonstration that rDNA flanking regions are conserved across taxa and that flanking regions of sufficient length are distinct within a genome allowed for the development of riboSeed, a \textit{de fere novo} assembly method. riboSeed  utilizes rDNA flanking regions to act as barcodes for repeated rDNAs, allowing the assembler to correctly place and orient the rDNA. \textit{de fere novo} assembly can improve the assembly by bridging across ribosomal regions, and, in cases where rDNA repeats would otherwise result in incomplete scaffolding, can result in closure of a draft genome when used in conjunction with existing polishing tools. Although riboSeed is far from a silver bullet to provide perfect assemblies from short read technology, it shows the utility of using genomic reference data and mixed assembly approaches to overcome algorithmic obstacles. This approach to resolving rDNA repeats may allow further insight to be gained from large public repositories of short read sequencing data, such as SRA, and when used in conjunction with other genome finishing techniques, provides an avenue towards genome closure.
