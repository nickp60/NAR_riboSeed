We demonstrate that regions flanking equivalent rDNAs from related strains show a high degree of conservation in related organisms. This allows us to correctly place rDNAs within a newly sequenced isolate, even in the absence of the resolution that would be provided by long read sequencing. Comparing the regions flanking rDNAs within a single genome, we observed that when considering sufficiently large flanking regions, flanking sequences show enough variability to differentiate each instance of the rDNAs. Taken together with the cross-taxon homology, this allows inference of the location (i.e. the flanking regions) of rDNAs, and the variability of these flanking regions within a genome enables unique identification of reads likely belonging to a specific cluster.

The extent of sequence similarity between the sequenced isolate and the reference influences \textit{de fere novo} assembly. If fewer than 80\% of reads map to the reference, resulting pseudocontigs are treated as ``untrusted'' contigs by SPAdes to prevent spurious joining of contigs. Figure \ref{fig:degen} shows that although one should preferentially use the closest complete reference available for optimal results, the subassembly method is robust against moderate discrepancies between the reference and sequenced isolate's flanking regions.

Strains possessing a single rDNA (such as \textit{Mycobacterium abscessus}) do not suffer from repeated region assembly issues. Similarly, the rRNA coding regions in some taxa (such as \textit{Thermus thermophilus} or \textit{Leptospira interrogans}, see Supplementary section ``Atypical rDNA operon structure'', Figures S6 and S7) are not organized into operons. Such genomes do not require correction with riboSeed.

The method of constructing pseudocontigs implemented by riboSeed relies on having a relevant reference sequence, where the rDNA regions act as ``bait'', fishing for reads that likely map specifically to that region. This makes riboSeed a valuable tool for assembly or reassembly of bacterial or archaeal strains (Tables S7 and S8) for which such a reference is available, but application to community ecology where one may be sequencing novel organisms from unsequenced genera will be limited by the requirement for such a reference genome.  Although we show this ``baiting'' method to be an effective way to partition the reads, a more robust method may be to use a probabilistic representation of equivalent rDNA regions for the sequenced taxon. By using a database of sequence profiles (e.g. hidden Markov Models) from homologous rDNAs in a taxon, the step of choosing a single most appropriate reference might be circumvented. For datasets where the choice of reference determines riboSeed's effectiveness, a probabilistic approach may improve performance.

Several checks are implemented after subassembly to ensure that resulting pseudocontigs are fit for inclusion in the next round in the next mapping/subassembly iteration or the final \textit{de fere novo} assembly. If a subassembly's longest contig is greater than 3x the particular pseudocontig length or shorter than 6kb (a conservative minimum length of a 16S, 23S, and 5S operon), this is taken to be a sign of poor parameter choice so the user is warned, and by default no further seedings will occur to avoid spurious assembly. Such an outcome can be indicative of any of several factors: improper clustering of operons; insufficient or extraneous flanking sequence; sub-optimal mapping; inappropriate choice of \textit{k}-mer length for subassembly; inappropriate reference; or other issues. If this occurs, we recommend testing the assembly with different \textit{k}-mers, changing the flanking length, or trying alternative reference genomes. Mapping depth of the rDNA regions is also reported for each iteration; a marked decrease in mapping depth may also be indicative of problems.

Many published genome finishing tools and approaches offer improvements when applied to suitable datasets, but none (including the approach presented in this paper) is able in isolation to resolve all bacterial genome assembly issues. One constraint on the performance of riboSeed is the quality of rRNA annotations in reference strains. Although it is impossible to concretely confirm \textit{in silico}, we (and others \cite{Mariano2016}) have found several reference genomes during the course of this study that we suspect have collapsed rDNA repeats. We recommend using a tool such as 16Stimator \cite{Perisin2016} or rrnDB \cite{Stoddard2014} to estimate number of 16S (and therefore rDNAs) prior to assembly, or \texttt{stack} to assess mapping depths after running \texttt{seed}.

As riboSeed relies on de Bruijn graph assembly, the results can be affected by assembler parameters. Care should be taken to explore appropriate settings, particularly in regard to read trimming approach, range of \textit{k}-mers, and error correction schemes.

One difficulty in determining the accuracy of rDNA counts in reference genome occurs because genome sequences are often released without publishing the reads used to produce the genome assembly. This practice is a major hindrance when attempting to perform coverage-based quality assessment, such as to infer the likelihood of collapsed rDNAs. While data transparency is expected for gene expression studies, that stance has not been universally adopted when publishing whole-genome sequencing results. To ensure the highest quality assemblies from historical data, we strongly recommend that researchers share raw reads.
