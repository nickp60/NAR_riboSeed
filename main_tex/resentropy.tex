The use of rDNA flanking sequences to uniquely identify and place rDNAs in their genomic context requires their flanking sequences to be distinct within the genome for each region. This is expected to be the case for nearly all prokaryotic genomes where rRNA coding sequences are structured as operons. We determined that a 1kb flanking region was sufficient to include differentiating sequence (Figure S2). To demonstrate this, rDNA and 1kb flanking regions were extracted from \textit{E. coli} Sakai \cite{Hayashi2001} (BA000007.2), a strain in which  rDNAs have been well characterized \cite{Ohnishi2000}. These regions were aligned with MAFFT \cite{Katoh2002}, and consensus depth and Shannon entropy calculated for each position in the alignment (Figure \ref{fig:entropy}a).

Figures \ref{fig:entropy}a and S4 show that within a single genome the regions flanking rDNAs are variable between operons. This enables unique placement of reads at the edges of rDNA coding sequences in their genomic context (i.e. there is not likely to be confusion between the placements of rDNA edges within a single genome).
In \textit{E. coli} MG1655 (NC\_000913.3), the first rDNA is located 363 bases downstream of \textit{gmhB} (locus tag b0200). Homologous rDNA regions were extracted from 25 randomly selected complete \textit{E. coli} chromosomes (Table S2).  We identified the 20kb region surrounding \textit{gmhB} in each of these genomes, then annotated and extracted the corresponding rDNA and flanking sequences. These sequences were aligned with MAFFT, and the Shannon entropies and consensus depth plotted (Figure \ref{fig:entropy}b).

Figure \ref{fig:entropy}b shows that equivalent \textit{E. coli} rDNAs, plus their flanking regions, are well-conserved across several related genomes. Assuming that individual rDNAs are monophyletic within a taxonomic group, short reads that can be uniquely placed on a related genome's rDNA as a reference template are also likely able to be uniquely-placed in the appropriate homologous rDNA of the genome to be assembled.

Taken together, when these two properties hold, this allows for unique placement of reads from homologous rDNA regions in the appropriate genomic context. These ``anchor points'' effectively reduce the number of branching possibilities in de Bruijn graph assembly for each individual rDNA, and thereby permit reconstruction of a complete balanced path through the full rDNA region.
