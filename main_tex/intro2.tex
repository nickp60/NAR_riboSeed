The Illumina entries in NCBI's Sequence Read Archive (SRA) \cite{Kodama2012a} outnumber all other technologies combined by about an order of magnitude (Table S1). Draft assemblies from these datasets have systematic problems common to short read datasets, including gaps in the scaffolds due to the difficulty of resolving assemblies of repeated regions \cite{Whiteford2005,Treangen2011}. By resolving repeated regions during the assembly process, it may be possible to improve existing assemblies, and therefore obtain additional sequence information from existing short read datasets in the SRA.



The most common repeated regions are those coding for ribosomal RNA operons (rDNAs). As ribosomes are essential for cell function, sequencing of the 16S ribosomal region is widely used to identify prokaryotes and explore microbial community dynamics \cite{Weisburg1991,Clarridge2004,Woese1990,Case2007}. This region is conserved within taxa, yet retains enough variability to act as a bacterial ``fingerprint'' to separate clades informatively. However, the 16S, 23S, and 5S ribosomal subunit coding regions are often present in multiple copies within a single prokaryotic genome, and commonly exhibit polymorphism \cite{Coenye2003,Moreno2002,Lukjancenko2010,Vetrovsky2013}. These long, inexactly repeated regions \cite{Alkan2011} are problematic for short-read genome assembly. As rDNAs are frequently used as a sequence marker for taxonomic classification, resolving their copy number and sequence diversity from short read collections where the assembled genome has collapsed several repeats into a single region could help improve reference databases, increasing the accuracy of community analysis. We present here an \textit{in silico} method, riboSeed, that capitalizes on the genomic conservation of rDNA and flanking sequence within a taxon to improve resolution of these difficult regions and provide a means to benefit from unexploited information in the SRA/ENA short read archives.


riboSeed is most similar in concept to GRabB, the method of Brankovics et al. \cite{Brankovics2016} for assembling mitochondrial and rDNA regions in eukaryotes. Both use targeted assembly, but GRabB does not make inferences about the number of rDNA clusters present in the genome, or take advantage of their genomic context. In riboSeed, genomic context is resolved by exploiting both the rDNA sequences and their flanking regions, harnessing unique characteristics of the broader rDNA region within a single genome to improve assembly.


The riboSeed algorithm proceeds from two observations: first, that although repeated rRNA coding sequences within a single genome are nearly identical, their flanking regions (that is, the neighboring locations within the genome) are distinct in that genome, and second, that the genomic contexts of equivalent rDNA sequences are also conserved within a taxonomic grouping (Figure S??). riboSeed uses only reads that map to rDNA regions from a reference genome, and is not affected by chromosomal rearrangements that occur outside the flanking regions immediately adjacent to each rRNA.


Briefly, riboSeed uses rDNA regions from a closely-related organism's genome to help generate rDNA cluster-specific ``pseudocontigs'' derived only from the input short reads, that are seeded into the raw short reads to generate a final assembly. We refer to this process in this work as \textit{de fere novo} (meaning ``starting from almost nothing'') assembly.
