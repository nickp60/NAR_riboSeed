To establish whether riboSeed performs as well with short reads obtained by sequencing a complete prokaryotic chromosome as with simulated reads, we attempted to assemble short reads from a published hybrid Illumina/PacBio sequencing project. The hybrid assembly using long reads was able to resolve rDNAs directly, and provides a benchmark against which to assess riboSeed performance in terms of: (i) bridging sequence correctly across rDNAs, and (ii) assembling rDNA sequence accurately within each cluster.


Sanjar, et al. published the genome sequence of \textit{Pseudomonas aeruginosa} BAMCPA07-48 (CP015377.1) \cite{Sanjar2016}, assembled from two libraries: ca. 270bp fragmented genomic DNA with 100bp paired-end reads sequenced on an Illumina HiSeq 4000 (SRR3500543), and long reads from PacBio RS II. The authors obtained a closed genome sequence by hybrid assembly. We ran the riboSeed pipeline on only the HiSeq dataset in order to compare \textit{de fere novo} assembly to the hybrid assembly and \textit{de novo} assembly of the same reads, using the related genome \textit{P. aeruginosa} ATCC 15692 (NZ\_CP017149.1) as a reference.

\textit{de fere novo} assembly correctly assembled across all 4 rDNA regions, whereas \textit{de novo} assembly failed to assemble any rDNA regions (Table \ref{table:assemblyresults}A).

Comparing the BAMCPA07-48 reference to the \textit{de fere novo} assembly, we found a total of 9 SNPs in the rDNA flanking regions (Table \ref{table:snps}). The same regions from the ATCC 15692 reference used in the \textit{de fere novo} assembly showed 108 SNPs compared to the BAMCPA07-48 isolate.  This demonstrates that this subassembly scheme successfuly recovers the correct sequence with remarkably few SNPs, despite a large number of differences between the reference and the sequenced isolate, and that the riboSeed method does not simply transpose the reference genome rDNA sequence into the new assembly.

Further, to assess how riboSeed's assembly would compare to supplying the whole reference as a trusted contig in SPAdes (a strategy not recommended by the SPAdes authors), we assembled the same reads with the \textit{P. aeruginosa} ATCC 15692 as a trusted contig, and compared results to \textit{de fere novo} and \textit{de novo} assemblies. \textit{de novo} assembly yielded the lowest error rates, and \textit{reference-based}  assembly yielded the longest contigs, but \textit{de fere novo} assembly exhibited very low error rates, the highest genome recovery fraction, and the lowest number of contigs (Table S5).


We find that the \textit{de fere novo} assembly using short reads performs better than \textit{de novo} assembly using short reads alone. Comparison of \textit{de fere novo} to hybrid assembly allows assessment of \textit{de fere novo} accuracy, and indicates that \textit{de fere novo} can recover the rDNA sequence correctly placed in their genomic context, with a low error rate.
