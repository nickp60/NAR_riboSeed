To investigate the effect of short read length on riboSeed assembly, pIRS \cite{Hu2012} was used to generate paired-end reads from the complete \textit{E. coli} MG1655 and \textit{K. pneumoniae} NTUH-K2044 genomes, simulating datasets at a range of read lengths most appropriate to the sequencing technology. In all cases, 300bp inserts with 10bp standard deviation and the built-in error profile were used. Coverage was simulated at 20x to emulate low coverage runs and at 50x to emulate coverage close to the optimized values determined by Miyamoto \cite{Miyamoto2014} and Desai \cite{Desai2013}. \textit{De fere novo} assembly was performed with riboSeed using \textit{E. coli} Sakai and \textit{K. pneumoniae} HS11286 as references, respectively, and the results were scored with \texttt{score} (Figure \ref{fig:simreads}).


At either 20x or 50x coverage, \textit{de novo} assembly was unable to resolve any rDNAs with any of the simulated read sets. \textit{de fere novo} assembly with riboSeed showed improvement to both the \textit{E. coli} and \textit{K. pneumoniae} assemblies. Increasing depth of coverage and read length improves rDNA assemblies.
