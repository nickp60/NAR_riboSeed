%%%%%%%%%%%%%%%%%%%%%%%%%%%%%%%%%%%%%%%%%%%%%%%%%%%%%%%%%%%%%%%%%%%%%%%%%%%%
% Nick Waters's super awesome template for bioRxiv and suppl submissions
% In memory of Henry, Hermione, and Angus.
% May there be many tasty snails in pufferfish heaven
%%%%%%%%%%%%%%%%%%%%%%%%%%%%%%%%%%%%%%%%%%%%%%%%%%%%%%%%%%%%%%%%%%%%%%%%%%%%
\documentclass[10pt]{article}

\usepackage{geometry}
\geometry{marginparwidth=1cm,a4paper,verbose,tmargin=1.5cm,bmargin=1.5cm,lmargin=1.5cm,rmargin=1.5cm,headheight=0cm,headsep=0cm,footskip=0cm}
\usepackage{cite}
% \usepackage{hyperref} % make the references clickable
\setlength\parindent{0pt} % set indent to zero
\setlength{\parskip}{1.2em} % give a bit of space between paragraphs

\bibliographystyle{plain} % citation and bib style
\usepackage{grffile} %for underscores in file names
\usepackage[ruled]{algorithm2e} % for typesetting the riboSeed algorithm
\setlength{\algotitleheightrule}{0pt}
\usepackage{lineno, color} % controls line numbering
\modulolinenumbers[5]  % number every 5th line
% format our line numbers, cause no one likes boring numbers
\renewcommand\linenumberfont{\normalfont\tiny\color{blue}}
\usepackage{booktabs}  % less yucky tables
% nice little thing for table spacing (thanks Markus Püschel)
\newcommand{\ra}[1]{\renewcommand{\arraystretch}{#1}}
\usepackage{multirow} % for some pretty tables with merged cells

% my lined-off abstract
\newenvironment{myabstract}{%
\begin{quote} \baselineskip14pt \rule{.89\textwidth}{1.5pt} \vskip .1cm}
{ \vskip -.10cm \rule{.89\textwidth}{1.5pt} \end{quote}}

% supplementary (essentialy just resets numbering)
\newcommand{\beginsupplement}{%
  \setcounter{table}{0}
  \renewcommand{\thetable}{S\arabic{table}}%
  \setcounter{figure}{0}
  \renewcommand{\thefigure}{S\arabic{figure}}%
}
% pretty thought break
\def \thoughtbr {\begin{center}\noindent\rule{.4\textwidth}{0.4pt}  {\raisebox{-.5ex}{$\sim$}}  \rule{.4\textwidth}{0.4pt}\end{center}}

% \usepackage{graphicx}
\graphicspath{ {../riboSeed/Waters_et_al_2017/ms/riboFigs/} }  %  set the path to the figures

\usepackage{titlesec} % for adjusting spacing after the section headers
\titlespacing\subsection{0pt}{12pt plus 4pt minus 2pt}{-11pt plus 0pt minus 0pt}
\titlespacing\subsubsection{0pt}{12pt plus 4pt minus 2pt}{-11pt plus 0pt minus 0pt}


%%%%%%%%%%%%%%%%%%%%%%%%%%%%%%%%%%%%%%%%%%%%%%%%%%
% this hack stolen from Stack overflow is a macro to make a \cmmidrule
% for each column.  I added the ".4" spacing to get this to look prettier
%  if you have 5 columns, instead of \midrule, you could use \cmidrules{5}
\makeatletter
\newtoks\MD@cmidrules
\newcommand{\cmidrules}[1]{%
  \noalign{%
    \global\MD@cmidrules={}%
    \toks@={\cmidrule(l{.3\tabcolsep}r{.3\tabcolsep})}%
    \count@=\z@
    \loop\ifnum\count@<#1\relax
      \advance\count@\@ne
      \edef\MD@temp{\the\toks@{\the\count@-\the\count@}}%
      \global\MD@cmidrules\expandafter{\the\expandafter\MD@cmidrules\MD@temp}%
    \repeat
  }%
  \the\MD@cmidrules
}
\makeatother
%%%%%%%%%%%%%%%%%%%%%%%%%%%%%%%%%%%%%%%%%%%%%%%%%%

%%%%  This allows us to set the table font size up top
\usepackage{etoolbox}
\BeforeBeginEnvironment{tabular}{\begin{center}\footnotesize}
\AfterEndEnvironment{tabular}{\end{center}}
%%%%

\def \coli#1 {\textit{E.~coli~#1~}} %  #lazy
\def \ttilde {\raisebox{-.5ex}\textasciitilde} % fix for ugly cmodern tilde 20160829

%  format captions
\usepackage[%
    font={small},
    labelfont=bf,
    format=hang,
    format=plain,
    margin=0pt,
    width=0.9\textwidth]{caption}
\usepackage[list=true]{subcaption}  % allow for subfigures with captions
\usepackage{amssymb}  % for checkmark symbols
\usepackage{graphicx}  % trim figures
\usepackage{colortbl}  % adjust table colors
\definecolor{lgray}{gray}{.90}  % line grey
\definecolor{tgray}{gray}{.40}  % text grey
\definecolor{egray}{gray}{.01}  % emphasis grey
\usepackage[yyyymmdd,hhmmss]{datetime} % for timestamp

\title{\textbf{Supplementary Data}\\ \fontseries{m}\selectfont riboSeed: leveraging prokaryotic genomic architecture to assemble across ribosomal regions}

\author
{\small Nicholas R. Waters,$^{1,2}$ Florence Abram,$^{1}$ Fiona Brennan,$^{1,3}$ Ashleigh Holmes,$^{4}$ and Leighton Pritchard$^{2\ast}$\\
\\
\normalsize{\textit {$^{1}$Department of Microbiology, School of Natural Sciences, National University of Ireland, Galway, Ireland}}\\
\normalsize{\textit {$^{2}$Information and Computational Sciences, James Hutton Institute, Invergowrie, Dundee DD2 5DA, Scotland}}\\
\normalsize{\textit {$^{3}$Soil and Environmental Microbiology, Environmental Research Centre, Johnstown Castle, Wexford, Ireland}}\\
\normalsize{\textit {$^{4}$Cell and Molecular Sciences, James Hutton Institute, Invergowrie, Dundee DD2 5DA, Scotland}}\\
\\
\footnotesize{$^\ast$To whom correspondence should be addressed: leighton.pritchard@hutton.ac.uk}
}
\date{}

\begin{document}
\baselineskip20pt  % set line spacing; double spacing is 2xfontsize, in this case 11
\maketitle
\vspace*{-1.5cm}
{\begin{center}\footnotesize\date{Compiled: \today ~\currenttime}\end{center}}

\beginsupplement
\section*{Extended Methods}

\subsection*{Reference Selection Recommendations}
Using a close reference sequence maximizes your chance for a successful assembly.

Here are two ways that can be used reference to select a reference for your isolate: a robust method, and a quick method.

\subsubsection{Method 1: Kraken}

Kraken is a kmer-based phylogeny tool that can be used to idenify the strains present in a metagenomic dataset;  the installation and usage instructions can be found here: \url{https://ccb.jhu.edu/software/kraken/}. Download and install Kraken, along with the MiniKraken database from their website. Run Kraken on the isolate's reads, and generate the Kraken report.

The MiniKraken database was built from all the complete genomes from RefSeq, allowing the user identify which strain in the database has the closest match to the sequenced isolate.

\subsubsection{Method #2: reads2type and  cgFind}
<https://nickp60.github.io/cgfind>`


reads2type is also a kmer-based phylogeny tool, but it relies on a lightweight, prebuilt database that allows the analysis to be performed in your web browser, and it doesn't require you to upload your whole read file to a webserver.  It works by taking one read at a time from your file, generating 55-mers, and comparing to its databse. If there is not enough taxanomic information to indentify the isolate off of that read alone, additional reads will be processed until a single taxonomy is achieved.  This method works best on trimmed reads. `Instructions and the webserver can be found here <https://cge.cbs.dtu.dk/services/Reads2Type/>`__

Once you have a genus and species, you can use ``cgfind``, a tool we developed to provide easy access to downloadable genomes based on the complete prokaryotic genomes found in NCBI.  `it can be found here <https://nickp60.github.io/cgfind>`__  Just enter your genus and species name, and select one of the available strains to download.



\subsection*{Making the artificial test genome}
The artificial genome used for testing was constructed using the \texttt{makeToyGenome.sh} script included in the GitHub repository under the \textbf{\texttt{scripts}} directory. Briefly, the 7 rDNA regions from the \coli{Sakai} genome were extracted with 5kb flanking sequence upstream and downstream; these sequences were then concatenated end to end to form a single, \ttilde100kb sequence containing the 7 rDNAs as well as their flanking context.


\subsection*{Effect of reference sequence identity on riboSeed performance}

The following range of substitutions were introduced into a artificial genome using the \texttt{runDegenerate.sh} script (included in the GitHub repository under the \textbf{\texttt{scripts}} directory), which facilitates the following procedure: 0.0, 0.0025, 0.0050, 0.0075, 0.0100, 0.0150, 0.0200, 0.0250, 0.0500, 0.0750, 0.1000, 0.1250, 0.1500, 0.1750, 0.2000, 0.2250, 0.2500, 0.2750, 0.3000. An artificial test genome is constructed (see above), and reads simulated using pIRS (100bp, 300bp inserts, stdev 10, 30-fold coverage, built-in error profile).  Then, for each of a range of substitution frequencies, substitutions are introduced into the simulated genome, either just in the flanking regions or throughout. riboSeed is run on the reads using the mutated genome as the reference, and the results are evaluated with riboScore. This script was run 100 times, using a different random seed each time.  As pseudo random number generation may differ between operating systems, comparable but not identical results can be expected.

\section*{Performance on Archaeal Data}

% \subsection*{}
We assessed the effectiveness of riboSeed with assembling archaeal genomes. Most (\ttilde55\%) archaeal genomes have only a single rDNA, and none has been observed to have more than four. As riboSeed requires a sequencing dataset and a reference genome, applicability was limited; of the 104 entries in \textit{rrn}DB with multiple rDNAs, only 7 had multiple entries at the species level. Among those, only 2 had publicly available short read data. We used riboSeed to re-assemble \textit{Methanosarcina barkeri Fusaro DSMZ804} (Ion Torrent PGM, 89bp single-end reads) and \textit{Methanobacterium formicicum st. BRM9} (Illumina HiSeq 2000, 100bp paired-end reads). \textit{Methanobacterium formicicum st. JCM10132} (DRR017790 ) and \textit{Methanosarcina barkeri Fusaro DSMZ804} (SRR2064286) were the only ones that were suitable for riboSeed, meaning that there was publicly available short read data and that there is a related genome at the species level which is complete.


\textit{M. formicicum st. JCM10132} was sequenced on an Ion Torrent PGM, generating 106.5Mbp of single-end data. \textit{M formicicum BRM9} (CP006933.1) was used as a reference. The resulting \textit{de fere novo} assembly resulted in assembly of 1 of 2 rDNA gaps. This represents the first application of riboSeed to Ion Torrent data.


\textit{Methanosarcina barkeri Fusaro DSMZ804} was sequenced using an Illumina HiSeq2000 with 101bp paired-end reads, with an average fragment length of 400bp. We downsampled to use 5\% of the 19.4Gbp dataset. \textit{Methanosarcina barkeri str. Wiesmoor} was used as a reference. The resulting riboSeed assembly showed correct assembly of 3 of 3 rDNAs, while \textit{de novo} assemble failed to resolve any.


Taken together, we show that given appropriate datasets, archaeal datasets can be processed in the same manner used for bacteria.

\section*{Key Parameters}
\subsection*{\texttt{--ref_as_contig}}
The assembly that results from including riboSeed's ``long reads'' is sensitive to the manner in which they are incorporated into the \textit{de novo} assembly. Here, for our analyses, we used the SPAdes assembler, as it has built-in ways to include contigs (using the ``--trusted-contigs'' or ``--untrusted-contigs'') in FASTA format.  Other assemblers could be used, but most require long reads to have a quality score associate with them, preventing direct use of riboSeeds long reads.

As mentioned in the Methods section, riboSeed uses the reference rDNA region in the initial subassembly;  in subsequent subassemblies, the longest contig of the previous subsassembly is used.  The manner in which these regions are can be one of four options to \texttt{--ref_as_contig}: \texttt{trusted}, \texttt{untrusted}, \texttt{inferr}, or \texttt{ignore}.  Additionally, if the user is worried that the reference rDNA will too heavily influence the initial subassembly, the can enable the \texttt{--initial_consensus} flag to use a mapping consensus assembly instead of the de Bruijn graph based assembly from SPAdes.

The default manner in which rDNA regions (either from the reference or from the previous iteration's subassembly) behaviour is to infer (\texttt{--ref_as_contig infer}): if the percent of reads mapping to  the (whole) reference sequence  is over 80\%, than the rDNA region will be included as a trusted contig.  If below 80\%, the reads will be treated as untrusted.

If a user wishes to have the subassemblies only using the reads (true \textit{de novo} assembly), they can use the \texttt{ignore} option.  We only recommend this with very close references.

Further, if the user wishes to explicity define the behaviour, \texttt{trusted} or \texttt{untrusted} can be provided to the \texttt{--ref_as_contig} argument.

\subsection*{\texttt{--}}




\thoughtbr
\newpage

\begin{table}[]
  \centering
  \caption{Hits resulting from searching the SRA database for various sequencing technologies as of January, 2017}
  \label{table:searchterms}
  \begin{tabular}{lrr}
    \toprule
    Search term & Hits & Percentage \\
    \midrule
    illumina & 2242225 & 94.27 \\
    pacbio & 21131 & 0.89 \\
    ion & 30560 & 1.28 \\
    roche & 42445 & 1.78 \\
    oxford & 12301 & 0.52 \\
    solid & 29791 & 1.25 \\
    \arrayrulecolor{lgray}\hline
    Total & 2378453 & 100\\
    \arrayrulecolor{black}
    \bottomrule
\end{tabular}
\end{table}

%%%%%%%%%%%%%%%%%%%%%%%%%%%%%%%%%%%%%%%%%%%%%%%%%%%%%%%%%%%%%%%%%%%%%%%
\begin{table}[!h]
\centering
\caption{Accessions for 25 \textit{E. coli} genomes}
\label{table:accessions}
\begin{tabular}{l}
  \toprule

  GCA\_000005845.2\_ASM584v2                    \\
  GCA\_000019385.1\_ASM1938v1                   \\
  GCA\_000026245.1\_ASM2624v1                   \\
  GCA\_000026345.1\_ASM2634v1                   \\
  GCA\_000026545.1\_ASM2654v1                   \\
  GCA\_000146735.1\_ASM14673v1                  \\
  GCA\_000257275.1\_ASM25727v1                  \\
  GCA\_000520055.1\_ASM52005v1                  \\
  GCA\_000732965.1\_ASM73296v1                  \\
  GCA\_001007915.1\_ASM100791v1                 \\
  GCA\_001442495.1\_ASM144249v1                 \\
  GCA\_001469815.1\_ASM146981v1                 \\
  GCA\_001660565.1\_ASM166056v1                 \\
  GCA\_001660585.1\_ASM166058v1                 \\
  GCA\_001753565.1\_ASM175356v1                 \\
  GCA\_001888075.1\_ASM188807v1                 \\
  GCA\_001901025.1\_ASM190102v1                 \\
  GCA\_001936315.1\_ASM193631v1                 \\
  GCA\_002056065.1\_ASM205606v1                 \\
  GCA\_002078295.1\_ASM207829v1                 \\
  GCA\_002156825.1\_ASM215682v1                 \\
  GCA\_002163935.1\_ASM216393v1                 \\
  GCA\_002192275.1\_ASM219227v1                 \\
  GCA\_002220265.1\_ASM222026v1                 \\
  GCA\_900096795.1\_Ecoli\_AG100\_Sample3\_Doxycycline\_Assembly
  \bottomrule
  {\tiny   All available at \texttt{ftp://ftp.ncbi.nlm.nih.gov/genomes/all/GCA/}}

\end{tabular}
\end{table}
%%%%%%%%%%%%%%%%%%%%%%%%%%%%%%%%%%%%%%%%%%%%%%%%%%%%%%%%%%%%%%%%%%%%%%%



\begin{table}[]
\centering
\caption{Strain names and accessions for reference genomes used in this study}
\label{table:strainlist}
\begin{tabular}{ll}
  \toprule
  Strain Name & Accession \\
  \midrule
  \textit{E. coli MG1655} & NC\_000913.3 \\
  \textit{A. hydrophila ATCC 7966} & NC\_008570.1 \\
  \textit{B. cereus ATCC 10987} & AE017194.1 \\
  \textit{B. cereus NC7401} & NC\_016771.1 \\
  \textit{B. fragilis 638R} & FQ312004.1 \\
  \textit{R. sphaeroides ATCC 17029} & NC\_009049.1, NC\_009050.1 \\
  \textit{S. aureus TCH1516} & NC\_010079.1 \\
  % \textit{S. aureus NCTC 8325} & NC\_007795.1 \\
  % \textit{S. aureus FDA209P} & AP014942.1 \\
  \textit{S. aureus MRSA252} & BX571856.1 \\
  \textit{V. cholerae El Tor str. N16961} & NC\_002505.1, NC\_002506.1 \\
  \textit{X. axonopodis pv. Citrumelo} & CP002914.1 \\
  \textit{P. aeruginosa BAMCPA07-48} & CP015377.1 \\
  \textit{P. aeruginosa ATCC 15692} & NZ\_CP017149.1\\
  \bottomrule

\end{tabular}
\end{table}

\begin{table}[]
  \centering
  \caption{Software Versions}
  \label{table:software}
  \begin{tabular}{ll}
    \toprule
    Tool & Version \\
    \midrule
    Mauve & 2015-02-13 build 0 \\
    BLAST+ & 2.2.28+ \\
    Barrnap & 0.8 \\
    BWA & 0.7.8-r455 \\
    samtools & 1.4.1 \\
    MAFFT & v7.310 \\
    SPAdes & v3.9.0 \\
    QUAST & 4.4 \\
    bedtools & 2.17.0 \\
    EMBOSS & 6.5.7 \\
    pIRS & 2.0.2\\
    \bottomrule
  \end{tabular}
\end{table}




%%%%%%%%%%%%%%%%%%%%%%%%%%%%%%%%%%%%%%%%%%%%%%%%%%%%
\pagebreak
%%%%%%%%%%%%%%%%%%%%%%%%%%%%%%%%%%%%%%%%%%%%%%%%
\SetKwProg{Scan}{riboScan}{ }{end}
\SetKwProg{Select}{riboSelect}{ }{end}
\SetKwProg{Seed}{riboSeed}{ }{end}
\begin{figure}[h]
  \centering
  \begin{minipage}{.6\linewidth}
    \begin{algorithm}[H]
      % \KwData{reference, reads}
      %  % initialization\;
      % \Scan{(reference fastas)}{
      % \For{chromosome in reference}{
      % run Barrnap to find rRNAs\;
      % add locus tags to gff\;
      % convert to genbank\;
      % }
      %   combine genbanks\;
      %   return scannedScaffolds\;
      % }
      %   \hrule
      % % \end{algorithm}
      % % \pagebreak
      % % \begin{algorithm}[H]
      %   \Select{(scannedScaffolds)}{
      %   parse coordinated of rRNAs\;
      %   cluster into operons\;
      %   return grouped\_loci\;
      % }
      %   \hrule
      % \end{algorithm}
      % \pagebreak
      % \begin{algorithm}[H]
      %   \Seed{(scannedScaffolds, grouped\_loci, reads, iters)}{
      %   reference = scannedScaffolds\;
      %   clusters = parse grouped\_loci\;
      %   \For{i in iters}{
      %   map reads to reference\;
      %   \For{cluster in clusters}{
      %   filter and extract reads within clusters and flanking\;
      %   subassemble with SPAdes\;
      %   return pseudocontig\;
      % }
      %   assess subassembly\;
      %   \If {success}{
      %   make fauxgenome from pseudocontigs \;
      %   reference = fauxgenome
      %   continue
      % }
      %   \If{last iteration}{
      %   run SPAdes with pseudocontigs\;
      % }
      % }
      % }
      \Seed{(reference, riboSelect\_clusters, reads, iters, flanking\_width)}{
        ref = reference\;
        clusters = parse riboSelect\_clusters\;
        region = clusters + flanking\_width\;
        \For{i in iters}{
          map reads to ref\;
          \For{cluster in clusters}{
            filter and extract reads region\;
            subassemble\;
            return pseudocontig\;
          }
          assess subassembly\;
          \If {success}{
            make pseudogenome from pseudocontigs \;
            ref = pseudogenome \;
          }
        }
        run assembler with reads and pseudocontigs\;
      }
    \end{algorithm}
  \end{minipage}
  \caption{Pseudocode of riboSeed algorithm}
  \label{fig:algo}
\end{figure}
%%%%%%%%%%%%%%%%%%%%%%%%%%%%%%%%%%%%%%%%%%%%%%%%%%%% 5


\begin{figure}[!h]
    \centering
    \hspace*{0cm}\includegraphics[width=.5\textwidth]{grouped_sakai_BLAST_results}
    \caption{BLASTn was used to perform \textit{in silico} DNA-DNA hybridization of all rDNA regions from \textit{E. coli Sakai} with variable flanking lengths. The number of hits is a proxy for occurrences in the genome; increasing the flanking length increases the specificity. (Points are jittered to aide visibility for overlapping values.)}
    \label{fig:blast}
  \end{figure}

\begin{figure}[!h]
    \centering
    \hspace*{0cm}\includegraphics[width=.60\textwidth]{simulated_genome}
    \caption{Assembly of artificial genome. \textit{De fere novo} results in closure of 3-5 rDNAs with the correct reference; only 1-2 rDNAs are correctly assembled using \textit{K. pneumoniae}.  No rDNAs are assembled with \textit{de novo} assembly. Scored with riboScore.py. N=8.}
    \label{fig:simgenome}
\end{figure}


\begin{figure}[h]
  \centering
  \begin{subfigure}[b]{.45\textwidth}
    \includegraphics[width=0.95\textwidth]{gage_entropy_figures/NC_000913.3_entropy_plot}
    \caption{\textit{E. coli MG1655} (NC\_000913.3)}
    \label{fig:ent_coli}
  \end{subfigure}
  \begin{subfigure}[b]{.45\textwidth}
    \includegraphics[width=0.95\textwidth]{gage_entropy_figures/CP003200.1_entropy_plot}
    \caption{\textit{K. pneumoniae subsp. pneumoniae HS11286} (CP003200.1) }
    \label{fig:ent_pneumo}
  \end{subfigure}
  \begin{subfigure}[b]{.45\textwidth}
    \includegraphics[width=0.95\textwidth]{gage_entropy_figures/NZ_CP017149.1_entropy_plot}
    \caption{\textit{P. aeruginosa strain ATCC 15692} (NZ\_CP017149.1)}
    \label{fig:ent_pao}
  \end{subfigure}
  \begin{subfigure}[b]{.45\textwidth}
    \includegraphics[width=0.95\textwidth]{gage_entropy_figures/NC_008570.1_entropy_plot}
    \caption{\textit{A. hydrophila ATCC 7966} (NC\_008570.1)}
    \label{fig:ent_aero}
  \end{subfigure}
  \begin{subfigure}[b]{.45\textwidth}
    \includegraphics[width=0.95\textwidth]{gage_entropy_figures/AE017194.1_entropy_plot}
    \caption{\textit{B. cereus ATCC 10987} (AE017194.1)}
    \label{fig:ent_cereus_atcc}
  \end{subfigure}
  \begin{subfigure}[b]{.45\textwidth}
    \includegraphics[width=0.95\textwidth]{gage_entropy_figures/NC_016771.1_entropy_plot}
    \caption{\textit{B. cereus NC7401} (NC\_016771.1)}
    \label{fig:ent_cereus_nc}
  \end{subfigure}
  \begin{subfigure}[b]{.45\textwidth}
    \includegraphics[width=0.95\textwidth]{gage_entropy_figures/FQ312004.1_entropy_plot}
    \caption{\textit{B. fragilis 638R} (FQ312004.1)}
    \label{fig:ent_frag}
  \end{subfigure}
  \begin{subfigure}[b]{.45\textwidth}
    \includegraphics[width=0.95\textwidth]{gage_entropy_figures/NC_009050.1_entropy_plot}
    \caption{\textit{R. sphaeroides  ATCC 17029} (NC\_009049.1, NC\_009050.1)}
    \label{fig:ent_rhodo}
  \end{subfigure}
\end{figure}
\begin{figure}
  \centering
  \ContinuedFloat
  \begin{subfigure}[b]{.45\textwidth}
    \includegraphics[width=0.95\textwidth]{gage_entropy_figures/NC_010079.1_entropy_plot}
    \caption{\textit{S. aureus TCH1516} (NC\_010079.1)}
    \label{fig:ent_tch}
  \end{subfigure}
  % \begin{subfigure}[b]{.45\textwidth}
  %   \includegraphics[width=0.95\textwidth]{gage_entropy_figures/NC_007795.1_entropy_plot}
  %   \caption{\textit{S. aureus NCTC 8325} (NC\_007795.1)}
  %   \label{fig:ent_nctc}
  % \end{subfigure}
  % \begin{subfigure}[b]{.45\textwidth}
  %   \includegraphics[width=0.95\textwidth]{gage_entropy_figures/AP014942.1_entropy_plot}
  % \caption{\textit{S. aureus FDA209P} (AP014942.1)}
  % \label{fig:entfda}
  % \end{subfigure}
  \begin{subfigure}[b]{.45\textwidth}
    \includegraphics[width=0.95\textwidth]{gage_entropy_figures/NC_002506.1_entropy_plot}
    \caption{\textit{V. cholerae El Tor str. N16961} (NC\_002505.1) (NC\_002506.1)}
    \label{fig:ent_vib}
  \end{subfigure}
  \begin{subfigure}[b]{.45\textwidth}
    \includegraphics[width=0.95\textwidth]{gage_entropy_figures/CP002914.1_entropy_plot}
    \caption{\textit{X. axonopodis pv. Citrumelo} (CP002914.1)}
  \end{subfigure}
  \caption{riboScan.py,riboSelect.py, and riboSnag.py were run on all the genomes used as references for \textit{de fere novo} assemblies. Consensus alignment depth (grey bars) and Shannon entropy (black points, smoothed entropy as red line) for aligned rDNA regions.}
  \label{fig:ent_gage}

\end{figure}

\end{document}
