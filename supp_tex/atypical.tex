\section*{Atypical rDNA operon structure}

Bacterial (and many archaeal) ribosomal RNA coding regions are commonly arranged into operons consisting of a 16S rRNA, 23S rRNA, and one or more 5S rRNAs, often with various tRNAs interspersed. In the course of this study, we observed some taxa lacking this typical 16S--23S--5S rRNA operon. When rDNAs are not structured into operons, assemblies from short reads do not suffer from the issue of long repeats and so do not require specialised approaches to assembly, such as riboSeed. We developed a module called \texttt{structure} for plotting rDNAs across a collection of genomes; this is available for riboSeed as of version 0.4.50.  Figure \ref{fig:atypical} show the operon arrangement of a few examples of organisms exhibiting atypical operon structure. For comparison, the rDNAs in the reference strains used in this study are shown in Figure \ref{fig:typical}.

\begin{figure}[H]
  \centering
  \hspace*{0cm}\includegraphics[width=1.0\textwidth]{odd_rDNA_relative_locations}
  \caption{Atypical rDNA operon structure in select taxa. rRNA lengths are not shown to scale. Note that the NCBI record for \textit{Helicobacter pylori} (NC\_000915.1) shows a 5S rRNA not detected by Barrnap. }
  \label{fig:atypical}
\end{figure}

\begin{figure}[H]
  \centering
  \hspace*{0cm}\includegraphics[width=1.0\textwidth]{normal_rDNA_relative_locations}
  \caption{Typical rDNA structure exhibited by strains used in this study.}
  \label{fig:typical}
\end{figure}
