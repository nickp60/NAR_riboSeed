\section*{Key Parameters}
\subsection*{\texttt{--ref\_as\_contig}}
The assembly that results from including riboSeed's ``long reads'' is sensitive to the manner in which they are incorporated into the \textit{de novo} assembly. Here, for our analyses, we used the SPAdes assembler \cite{Bankevich2012}, as it has built-in ways to include contigs (using the ``--trusted-contigs'' or ``--untrusted-contigs'') in FASTA format.  Other assemblers could be used, but most require long reads to have a quality score associated with them, preventing direct use of riboSeed's long reads.

As mentioned in the Methods section, riboSeed uses the reference rDNA region in the initial subassembly;  in subsequent subassemblies, the longest contig of the previous subsassembly is used.  These regions can be treated one of four ways using the \texttt{--ref\_as\_contig} argument: \texttt{trusted}, \texttt{untrusted}, \texttt{infer}, or \texttt{ignore}.  Additionally, if the user is worried that the reference rDNA will too heavily influence the initial subassembly, they can enable the \texttt{--initial\_consensus} flag to use a mapping consensus assembly instead of the de Bruijn graph based assembly from SPAdes.

The default manner in which rDNA regions (either from the reference or from the previous iteration's subassembly) behaviour is to infer (\texttt{--ref\_as\_contig infer}): if the percent of reads mapping to  the (whole) reference sequence  is over 80\%, than the rDNA region will be included as a trusted contig.  If below 80\%, the reads will be treated as untrusted.

If a user wishes to have the subassemblies only using the reads (true \textit{de novo} assembly), they can use the \texttt{ignore} option.  We only recommend this with very close references.

Further, if the user wishes to explicitly define the behaviour, \texttt{trusted} or \texttt{untrusted} can be provided to the \texttt{--ref\_as\_contig} argument.

\subsection*{\texttt{--score\_min}}
By default, the accepted alignment score for BWA mapping is $\frac{1}{2}$ the read length.  If needed, this can be increased for greater stringency when dealing with more divergent references, or decreased to include more reads, which may be advantageous when assembling a low coverage dataset.
